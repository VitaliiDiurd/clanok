\documentclass[10pt,twoside,slovak,a4paper]{coursepaper}
\usepackage[slovak]{babel}
%\usepackage[T1]{fontenc}
\usepackage[IL2]{fontenc}
\usepackage[utf8]{inputenc}
\usepackage{graphicx}
\graphicspath{ {./images/} }
\usepackage{url}
\usepackage{hyperref}


\usepackage{cite}
%\usepackage{times}

\pagestyle{headings}

\title{Recommender systems for YouTube\thanks{Semestrálny projekt v predmete Metódy inžinierskej práce, ak. rok 2024/25, vedenie: Diurd Vitalii}} 

\author{Diurd Vitalii\\[2pt]
	{\small Slovenská technická univerzita v Bratislave}\\
	{\small Fakulta informatiky a informačných technológií}\\
	{\small \texttt{xdiurd@stuba.sk}}
	}

\date{\small october 2024} % upravte

\begin{document}

\maketitle

\begin{abstract}
This article will contain information about YouTube recommender system, the way it works and why it has such a crucial role in enhancing user experience. I will explore the way YouTube`s recommender system operates, focusing on core algorithms and real-world applications. The primary objective is to understand how recommendation models (collaborative filtering, content-based filtering, hybrid approaches, etc.) work together to give user content based on his wishes. The other point of this article is to investigate how user`s behavior, including videos, comments, likes and other video interactions, helps system to understand each individual`s preferences, keeping users engaged on the platform for longer periods of time. The algorithm system not only benefits user by helping him in discovering something new, but also supports video creators by promoting their videos to proper audience, increasing interactions and viewer retention. I aim to show, by example of YouTube,   how different services recommender system is a powerful engine that helps both content discovery and platform success. I will also try to explain the importance of recommendations in nowadays digital world. The main source of information for this article was «Deep Neural Networks for YouTube Recommendations» [13].  Written by Paul Covington, Jay Adams and Emre Sargin. This article contains a big amount of information about YouTube`s algorithms and recommendation systems in general. Other sources will also be used in writing the article, all links are going to be included in used materials.
\end{abstract}



\section{Introduction} \label{intro}
\section{Core Algorithms in YouTube’s Recommender System} \label{core}
\subsection{Collaborative Filtering} \label{core:collaborative}
\subsection{Content-Based Filtering} \label{core:cont-based}
\subsection{Hybrid Approaches:} \label{core:hybrid}

\section{User Behavior and Engagement in Recommendations} \label{behavior}
\subsection{User Data and Its Role} \label{behavior:userdata}
\subsection{Engagement Maximization} \label{behavior:engagement}
\subsection{Content Creator Support} \label{behavior:ccsuport}
\paragraph{Veľmi dôležitá poznámka.}
\section{Architecture and Deep Learning Models in YouTube Recommendations} \label{arch}
\subsection{Candidate Generation Model} \label{arch:cangen}
\subsection{Ranking Model} \label{arch:rank}
\subsection{Deep Neural Networks and TensorFlow} \label{arch:dnnt}
\section{Conclusion} \label{dolezitejsia}

\bibliography{literatura}
\bibliographystyle{alpha}

\end{document}
